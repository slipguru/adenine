\documentclass[twoside,11pt]{article}

% Any additional packages needed should be included after jmlr2e.
% Note that jmlr2e.sty includes epsfig, amssymb, natbib and graphicx,
% and defines many common macros, such as 'proof' and 'example'.
%
% It also sets the bibliographystyle to plainnat; for more information on
% natbib citation styles, see the natbib documentation, a copy of which
% is archived at http://www.jmlr.org/format/natbib.pdf

\usepackage{jmlr2e}
\usepackage{multirow}
\usepackage{booktabs}
\usepackage{xspace}
\usepackage[table]{xcolor}
\usepackage{graphicx}
\usepackage[inline]{enumitem}
% \usepackage{siunitx}
% Definitions of handy macros can go here

\newcommand{\dataset}{{\cal D}}
\newcommand{\fracpartial}[2]{\frac{\partial #1}{\partial  #2}}

\newcommand{\ade}{\texttt{adenine}\@\xspace}
\newcommand*{\ie}{i.e.\@\xspace}
\newcommand*{\eg}{e.g.\@\xspace}
\newcommand{\todo}[1]{\textcolor{red}{{\bf \{#1\}}}} %TODO

% Heading arguments are {volume}{year}{pages}{submitted}{published}{author-full-names}
\jmlrheading{X}{2016}{X-XX}{X/XX}{XX/XX}{Samuele Fiorini, Federico Tomasi and Annalisa Barla}

% Short headings should be running head and authors last names

\ShortHeadings{ADENINE}{Fiorini and Tomasi}
\firstpageno{1}

\begin{document}

\title{ADENINE --- A Data ExploratioN pIpeliNE}

\author{\name{Samuele Fiorini} \email{samuele.fiorini@dibris.unige.it}\\
\name{Federico Tomasi} \email{federico.tomasi@dibris.unige.it}\\
\name{Annalisa Barla} \email{annalisa.barla@unige.it}\\[1em]
\addr Department of Informatics, Bioengineering, \\Robotics and System Engineering (DIBRIS)\\
     University of Genoa\\
     Genoa, I-16146, Italy}


\editor{Editor name}

\maketitle

\begin{abstract}
Abstract here.
\end{abstract}

\begin{keywords}
Exploratory data analysis, unsupervised learning, dimensionality reduction, clustering
\end{keywords}

\section{Introduction}\label{sec:intro}


\section{Implementation}\label{sec:implem}
From an implemetative standpoint, \ade is built upon the concept of \emph{pipeline}, that is a sequence of four fundamental steps:
\begin{enumerate*}[label=(\roman*)]
  \item missing values imputing,
  \item data preprocessing,
  \item dimensionality reduction and
  \item clustering
\end{enumerate*} (see Figure~\ref{fig:workflow}).

\begin{figure}[h!]
  \caption{\ade workflow}\label{fig:workflow}
\end{figure}

For each step, a fair number of off-the-shelf algorithm implementations are available (see Table~\ref{sec:implem}). The vast majority of such implementations is inherited, or extended from \texttt{scikit-learn} \citep{scikit-learn}, a collection of machine learning tools implemented in \texttt{Python}. For the first step, \ade offers an extended version of the \texttt{sklearn.preprocessing.Imputer} class that adds the \emph{KNN} imputing method to the pre-existent features-wise \emph{mean}, \emph{median} and \emph{most frequent} choices.

In order to obtain exploratory analysis of fairly big datasets, \ade exploits the use of parallel computing in several ways. Each pipeline is designed to be completely independent from each other

\section{Experiments and results}
To assess the quality of the obtained results, we tested \ade on a set of synthetic and real dataset.

\todo{parla qui dei test synth}
\todo{TGCA}

\section{Conclusions}



%%%%%%%%%%%%%%%%%%%%%%%%%%%%%%%%%%%%%%%%%%%%%%%%%%%%%%%%%%%%%%%%%%%%%%%%%%%%%%%%%%%

\begin{table}[hbtp]
  {\caption{Pipelines building blocks and relative references (which are not reported when the definition is given in Section~\ref{sec:implem}).}}

  {\begin{tabular}{lll}
  \toprule
  \bfseries Step &   \bfseries Algorithms & \bfseries Ref.\\

  \multirow{2}{*}{Imputing} & Mean &  \\
  & Median & \\
  & KNN & \citep{troyanskaya2001missing} \\
  \midrule

  \multirow{4}{*}{Preprocessing} & Recentering &  \\
  & Standardize &  \\
  & Normalize &  \\
  & MinMax &  \\
  \midrule

  \multirow{9}{*}{\begin{tabular}{@{}c@{}}Dimensionality \\ reduction\end{tabular}} & Principal component Analysis (PCA) & \citep{jolliffe2002principal} \\
  & Incremental PCA & \citep{ross2008incremental} \\
  & Randomized PCA & \citep{halko2011finding} \\
  & Kernel PCA & \citep{scholkopf1997kernel} \\
  & Isomap & \citep{tenenbaum2000global} \\
  & Locally linear embedding & \citep{roweis2000nonlinear} \\
  & Spectral embedding & \citep{ng2002spectral} \\
  & Multidimensional scaling & \citep{borg2005modern} \\
  & \begin{tabular}{@{}l@{}}t-Distributed Stochastic \\ Neighbor Embedding (t-SNE)\end{tabular}   & \citep{van2008visualizing} \\
  \midrule

  \multirow{5}{*}{Clustering} & K-means &  \citep{bishop2006pattern}\\
  & Affinity propagation & \citep{frey2007clustering} \\
  & Mean Shift & \citep{comaniciu2002mean} \\
  & Spectral & \citep{shi2000normalized} \\
  & Hierarchical & \citep{friedman2001elements} \\

  \bottomrule
  \end{tabular}}
\end{table}

% \acks{Acknowledgements go here.}

\vskip 0.2in
\bibliography{adenine}

\end{document}
